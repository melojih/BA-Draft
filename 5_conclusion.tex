
Long after the passage of anti-discrimination laws in the housing sector, pervasive disparities remain between minority and white landlords. In the absence of an experiment that randomizes property ownership, it is difficult for economists to measure the extent of discrimination empirically. I calculate the cost of discrimination to minority hosts by estimating the minority-white price disparity in the short-term housing market of Airbnb. Since prices only matter to hosts to the extent that they affect revenue, in this section I also estimate the impact of the price disparity on hosts' annual revenue.

In this paper, I estimate that minority hosts on Airbnb, especially black and Asian hosts, earn \$5-\$9 less per day for the same type of listing as white hosts. This amounts to a yearly loss in revenue of \$100 - \$300. Importantly, I rule out several alternative hypotheses that could be driving these results, and show that racial discrimination is the explanation most consistent with my findings. ---> MOVE THIS UP. 

To estimate the revenue loss that would result from the price differences found in the previous section, I construct a measure of revenue equal to: 

\[Total \: Revenue \ = \ Price \: Per \: Day \ * \ Reviews \: Per \: Month \ * \ 12\] 
%% \ and \: and \, and \quad and \qquad symbols make spaces in math mode

The estimates of the effect of host race on host revenue are in Table 7.%
	\footnote{I acknowledge that there are potential problems associated with using number of reviews as a proxy for demand, which are briefly mentioned in the introduction. I fully discuss them in Section 5.} 
Consistent with my prediction, all of the estimates are negative and in the range of \$100-300 dollars. The biggest yearly revenue loss in the entire sample is for black females at \$300, or about a 12\% loss. Black males and Asian women lose about \$160-180 throughout the year. Notably, white females, who have no statistically significant effects on their price in Table 3, have a significant \$144 loss in revenue. This is explained by the fact that they have lower number of reviews. ***This indicates that even hosts which may not have a price disparity might still be losing out in the Airbnb market. 

Overall, however, the same groups which had significantly lower prices also have lower revenues - black males, black females, Asian females, and white females all have significant effects in the range of several hundreds of dollars. 

Discrimination on the platform is pertinent as subject of research because Airbnb itself can do much to address these issues. In response to media outcry about allegations of discrimination, Airbnb updated its Discrimination Policy in September 2016, increasing instant bookings (the opportunity for guests to book without waiting for host approval) and making host profile pictures smaller. Evaluating Airbnb's efforts to address discrimination is therefore a relevant extension of this research. Since InsideAirbnb.com is continually being updated, there is now data available from webscrapes of listings after Airbnb's new discrimination policy took effect in September 2016. Future work can explore whether the policy helped curb discrimination on the platform by measuring the extent of discrimination before and after the policy took effect. If user interface really does influence the extent of discrimination in Airbnb, then the prices of minority and white hosts should start converging. This would suggest that discrimination was a cause of the price disparities in the first place. 

