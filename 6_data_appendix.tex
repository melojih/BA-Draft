
Inside Airbnb also provides some time-series information on prices, but since the each listing's price was not scraped daily, there are often week-long or month-long gaps in the time-series price data. A cursory glance at the time-series prices reveals that hosts do not change prices often, and if they do, they often reflect predictable weekend or holiday seasonality. There is therefore reason to believe that the prices posted at the time of the scrape are representative of a listing's price throughout the year. Because of the incompleteness of the time-series data set, I focus on the cross-sectional data for the main analysis.  

The data set does not include Airbnb's original neighborhood designations ``due to inaccuracies". Instead, the scraper assigned neighborhoods to each listing by comparing the geographic coordinates of the listing with each city's neighborhood designations.% 
	\footnote{Location information for listings is anonymized by Airbnb, and no exact address is provided for any listing. The location for a listing could be 0-150 meters from the actual address.} 
Figure 6 presents a map of Chicago's neighborhoods to give the reader a sense of the granularity of the neighborhood controls. 

Airbnb does not provide the demographic information of their users, so research assistants manually coded the hosts' demographic information. Research assistants were provided a link to the host profile picture and host name, and coded each picture according to the host's sex, race, and age. Only hosts with single-person pictures who were identifiably white, black, Asian, or Hispanic were included in the main analysis.%
	\footnote{All other types of profile pictures, including couples, groups of more than two people, children, pictures without a human face, or hosts of ambiguous race were dropped from the main analysis.}
Importantly, listings that no longer existed at the time of coding were also excluded.%
	\footnote{If certain groups of hosts systematically exited the Airbnb market between the time of the scrape and the time of the coding, dropping those listings could bias the results. Unfortunately, there is no way to verify the demographics of the hosts who dropped out, since Airbnb takes down their profile picture.}

Each RA was compensated based on the quantity of the listings they coded. This could create the incentive to code for speed rather than accuracy, so a simple double-checking process was put in place to check codings. For hosts whose picture was ambiguous on any of the dimensions of race, sex, or age, RAs were instructed to flag the listing. I subsequently coded each flagged picture and checked RA work. Due to manpower constraints, one RA coded each picture.%
	\footnote{It is important to note that the coding need not reflect the actual demographics of the host. Rather, it is sufficient that they are coded with the race, sex, and age that the average user on Airbnb would assume after looking at the profile picture. However, one limitation of this method is that the average University of Chicago undergraduate might not be representative of the average guest on Airbnb. With more resources, a more rigorous coding process could have been conducted. In future research, it would be preferable for two people to code each picture, and a third person to mediate any disagreement.}

\newpage	
{
\def\sym#1{\ifmmode^{#1}\else\(^{#1}\)\fi}
\begin{longtable}{l*{4}{c}}
\caption{Coding categories}\\
\hline\hline\endfirsthead\hline\endhead\hline\endfoot\endlastfoot
                    &\multicolumn{1}{c}{Sex}&\multicolumn{1}{c}{Race}&\multicolumn{1}{c}{Age}\\
\hline

1          &           Male         &           White         &           Young      \\ 
                    &                 &                  &          ($<$ 30)      \\
[1em]
2        &      Female  &      Black  &       Middle-aged     \\
                    &              &              &         \\
[1em]
3    &       Two males        &      Hispanic &       Old    \\
                    &              &              &     ($>$65)       \\
[1em]
4          &      Two females        &      Asian         &     Unknown      \\
                    &              &              &          \\
[1em]
5        &      Two people, different sex         &      Multiracial         &       \\
                    &              &              &       \\
[1em]
6    &       Unknown        &       Unknown  &        \\
                    &             &              &             \\
\hline\hline

\caption*{Notes: This table presents the categories according to which Research Assistants coded the race, sex, and age of the hosts and reviewers. Each host was assigned one category from each column. White refers only to non-Hispanic Whites. The Unknown categories are for profile pictures that are non-human, had more than one person, had only children, or did not contain a face. Multiracial is for pictures with two people of different race. For my main results, only Male/Female and White/Black/Hispanic/Asian were included, as interactions.}
\label{Table 1}


\end{longtable}
}
